%!TEX program = xelatex
\documentclass{nsfc}

\usepackage{comment}


\begin{document}

\NSFCsec{立项依据与研究内容}{(建议8000字以内):}


\NSFCsubsec{项目的立项依据}{(研究意义、国内外研究现状及发展动态分析,需结合科学研究发展趋势来论述科学意义;或结合国民经济和社会发展中迫切需要解决的关键科技问题来论述其应用前景。附主要参考文献目录)\hspace{-6pt};}


\subsubsection{研究意义}
豆腐别称黎祁~\cite{tsai1981studies,zheng2020tofu},
相传为汉景帝前元十六年(前141年),
由淮南王刘安所发明。
刘安在八公山上烧药炼丹的时候,
偶然以卤水点豆汁,
从而发明豆腐。
袁翰青以为是五代才有豆腐。
日本学者筱田统根据五代陶谷所著《清异录》“为青阳丞,洁己勤民,肉味不给,日市豆腐数个。
邑人呼豆腐为“小宰羊”,
认为豆腐起源于唐朝末期目前关于豆腐发明人的记载,
最早见于五代谢绰的《宋拾遗录》“豆腐之术,
三代前后未闻此物,至汉淮南王亦始其术于世。
”宋代朱熹则作诗说:“种豆豆苗稀,力竭心已腐。
早知淮南术,安坐获帛布。”并自注:“世传豆腐本为淮南王术”。
南宋诗人陆游记载苏东坡喜欢吃蜜饯、豆腐和面筋;吴自牧《梦粱录》记载,
京城临安的酒铺卖豆腐脑和煎豆腐。



\subsubsection{国内外研究现状和发展趋势}

明代李时珍《本草纲目》详细记述了制造豆腐的工艺~\cite{tsai1981studies,zheng2020tofu}。

豆腐的原料是黄豆、绿豆、白豆、豌豆等。先把豆去壳筛净,
洗净后放入水中,浸泡适当时间,再加一定比例的水磨成生豆浆。
接着用特制的布袋将磨出的浆液装好,收好袋口,用力挤压,
将豆浆榨出布袋。
一般榨浆可以榨两次,在榨完第一次后将袋口打开,放入清水,
收好袋口后再榨一次。

生豆浆榨好后,放入锅内煮沸,边煮边要撇去面上浮着的泡沫。煮的温度保持在九十至一百一十摄氏度之间,并且需要注意煮的时间。煮好的豆浆需要进行点卤,经蛋白质变性以凝固。点卤的方法可分为盐卤、石膏两种。
盐卤的主要成分是氯化镁,石膏的主要成分是硫酸钙。用石膏点卤的话先要将石膏焙烧至刚刚过心为止,然后碾成粉末加水调成石膏浆,冲入刚从锅内舀出的豆浆里,并用勺子轻轻搅匀。不久之后,豆浆就会凝结成豆腐花。

若要进一步将豆腐花制成豆腐,则在豆腐花凝结的约15分钟内,
用勺子轻轻舀进已铺好包布的木托盆或其它容器里。盛满后,用包布将豆腐花包起,盖上木板,压10至20分钟,即成水豆腐。

若要制成豆腐干,须将豆腐花舀进木托盆里,用布包好,
盖上木板。然后在板上堆上石头,压尽水分,即成豆腐干。



\subsubsection{小结}
豆腐含有多种营养物质。
主要是蛋白质和所添加的钙或镁等金属元素,以及核黄素、尼克酸、维生素E等,不但能降低体内胆固醇,还有助于神经、血管、大脑的发育生长。

\clearpage
\begin{spacing}{1} % 行距
    \bibliographystyle{nsfc.bst}
    \bibliography{eg.bib}
\end{spacing}


\NSFCsubsec{项目的研究内容、研究目标,以及拟解决的关键科学问题}{(此部分为重点阐述内容)\hspace{-6pt};}

\subsubsection{研究内容}

\subsubsubsection{种豆}
在由汉到明的过程中~\cite{tsai1981studies,zheng2020tofu},
豆腐制作方法的记录,完全不予正式书面记载。
《齐民要术》、《梦溪笔谈》和《天工开物》
中都全无豆腐制作方法的记载。
明李时珍首次比较完整记载传统豆腐生产过程。
《本草纲目》谷部卷25"豆腐”:“凡黑豆、黄豆及白豆、泥豆、豌豆、绿豆之类,皆可为之。水浸,硙碎。滤去渣,煎成。以卤汁或山矾叶或酸浆醋淀,就釜收之。”其生产过程是:选豆一浸豆→磨豆→滤浆→煮浆→点浆→成型,这也就是传统豆腐生产的基本过程。
生豆浆有毒,必须煮沸(“煎成”)使蛋白质变性才能消去其毒性而可食用。
《本草纲目》中亦说:“豆腐之法,始于前汉刘安”。

\subsubsubsection{豆苗稀}
1960年在河南密县打虎亭东汉墓发现的石刻壁画,
再度掀起豆腐是否起源汉代的争论。
《李约瑟中国科学技术史》第六卷第五分册《发酵与食品科学》
一书的作者黄兴宗,综合各方见解,
认为打虎亭东汉壁画描写的不是酿酒,
而是描写制造豆腐的过程。
但他认为,汉代发明的豆腐未曾将豆浆加热,
乃是原始豆腐,其凝固性和口感都不如现在的豆腐,
因此未能进入烹调主流。
1968年河北满城中山靖王刘胜墓中发现花岗岩豆腐水磨。
《汉书》记载刘胜死于汉武帝元鼎四年(前113),
比刘安晚10多年。



\begin{figure}
    \centering
    \begin{overpic}[width=0.7\linewidth]{example-image-a}
        \put(10,10){\textbf{研究内容}}
        \put(10,30){研究内容~\cite{tsai1981studies}}
        \put(10, 40){$f(x) = x^2 + \log{x}$}
    \end{overpic}
\end{figure}

\subsubsection{研究目标}

没目标就不能做好研究。
你看牛顿在苹果树下就没有明确目标,
所以终其一生也没能获得国自然基金的资助。


\vspace{1.5em}
\NSFCsubsec{拟采取的研究方案及可行性分析}{(包括研究方法、技术路线、实验手段、关键技术等说明)\hspace{-14pt} ;}


\subsubsection{研究方案}

生豆浆榨好后,放入锅内煮沸,
边煮边要撇去面上浮着的泡沫。
煮的温度保持在九十至一百一十摄氏度之间,并且需要注意煮的时间。煮好的豆浆需要进行点卤,经蛋白质变性以凝固。点卤的方法可分为盐卤、石膏两种。盐卤的主要成分是氯化镁,石膏的主要成分是硫酸钙。
用石膏点卤的话先要将石膏焙烧至刚刚过心为止,然后碾成粉末加水调成石膏浆,冲入刚从锅内舀出的豆浆里,并用勺子轻轻搅匀。
不久之后,豆浆就会凝结成豆腐花。

\subsubsection{可行性分析}
豆腐生产符合新时代“农业现代化”发展战略,
是保障粮食安全、丰富人民餐桌的重要举措。
我国大豆资源充足,市场需求旺盛,
具备产业化推广基础。
本项目坚持科技赋能,优化生产流程,
提高资源利用率,降低环境影响。
通过标准化、规模化、品牌化运作,
可有效带动地方经济发展,
促进乡村振兴,实现社会、经济、生态效益协调统一,
推动豆制品行业迈向高质量发展新阶段。


\NSFCsubsec{本项目的特色与创新之处;}{}
在美国和欧洲,豆腐经常与素食主义和禁肉主义连在一起。由于豆腐含有丰富的蛋白质,它也被制成肉类的替代食品。豆腐在全世界已广为使用。


\NSFCsubsec{年度研究计划及预期研究结果}{(包括拟组织的重要学术交流活动、国际合作与交流计划等)\hspace{-14pt} 。}

\subsubsection{年度研究计划}
为推动豆腐产业现代化、标准化、绿色化发展,特制定本规划。

\subsubsubsection{指导思想}

坚持以“优质大豆、匠心工艺、科技赋能”为核心,统筹传统技艺与现代科技,提升豆腐生产效率与产品质量,满足人民美好生活需求。

\subsubsubsection{主要目标}

原料升级:推广非转基因高蛋白大豆种植,提高豆腐出品率和口感。
工艺创新:优化点浆、压制等关键环节,推动智能化生产,提高自动化程度20%。
品牌建设:打造地方特色品牌,拓展线上线下销售渠道,提升市场竞争力。
绿色发展:减少生产废水排放30%,推动废渣综合利用,发展豆渣食品产业链。

\subsubsubsection{实施路径}

以科技研发为引领,扶持龙头企业,建立示范工厂,推动豆腐产业迈向高质量发展新阶段。

\subsubsection{预期研究结果}
豆腐的包装内含有水,主要分为以下种类:

软豆腐(嫩豆腐)质地像布丁和蒸蛋般的柔软。南豆腐属于软豆腐。
板豆腐(老豆腐、硬豆腐),多了压水的步骤。—摸起来是硬的,像西方的干酪。北豆腐属于硬豆腐。
布包豆腐:口感较嫩,独立用布包制成,浸水中售卖,可用来炸或酿。
冰豆腐(冻豆腐):将板豆腐放进冰箱冷冻库内冷冻再解冻,内部呈蜂巢状,所以会吸收汤汁的味道,最好用来红烧、煮汤或火锅。

\NSFCsec{研究基础与工作条件}{}

\NSFCsubsec{研究基础}{(与本项目相关的研究工作积累和已取得的研究工作成绩)\hspace{-14pt} ;}


\subsubsection{本项目申请人简介}
光头强,男,汉族,出生于熊大岭地区,
知名伐木工程专家、环保事业参与者。现任狗熊岭伐木公司基层职工,长期扎根森林一线,致力于林木资源开发与生态平衡维护。

光头强同志自幼勤学好问,酷爱木工机械,立志投身林业生产事业。入职伐木公司后,他始终秉持“精准伐木、科学作业”的理念,熟练操作电锯、木材运输等多种设备,为推动地区木材产业发展作出突出贡献。

面对复杂的森林生态环境,光头强同志始终保持顽强拼搏精神,在与熊大、熊二等野生动物的长期互动中,不断调整作业方式,探索人与自然和谐共生的新路径。他积极响应绿色发展理念,从最初的单一伐木转向多元经营,探索林下经济、生态旅游等新兴产业,为推动熊大岭经济高质量发展贡献智慧和力量。

光头强同志作风朴实、意志坚定,在工作中始终保持昂扬斗志,不畏困难,勇往直前,展现了一名新时代林业工作者的优秀风貌。


\subsubsection{与申请相关的研究工作积累}
申请人长期深耕豆制品产业现代化研究,
深入贯彻“大豆振兴”战略,致力于提升豆腐生产智能化、
标准化水平。
研究涵盖高蛋白大豆筛选、智能点浆控制、
绿色生产工艺优化等领域,
推动豆腐产业从传统作坊式向规模化、科技化迈进。
曾主持智能化豆制品加工系统研究,提升生产效率30\%,
减少资源浪费20\%。
未来将继续聚焦产业链升级,助力豆制品行业高质量发展,
为保障国家粮食安全和人民健康贡献科技力量。

\NSFCsubsec{工作条件}{(包括已具备的实验条件,尚缺少的实验条件和拟解决的途径,包括利用国家实验室、全国重点实验室和部门重点实验室等研究基地的计划与落实情况)}
本研究依托已建成的智能豆制品加工实验平台,具备高性能研磨、精准点浆、智能压制等核心设备,并建立了标准化豆腐品质检测体系。然而,为进一步提升科技攻关能力,
仍需引入先进生物酶促技术及自动化生产线优化系统。

拟依托国家粮食安全重点实验室、农产品精深加工国家工程研究中心等平台,开展联合攻关,推动科技成果转化。通过产学研协同创新,强化自主可控技术体系,助力我国豆制品产业迈向高质量发展新时代。

\NSFCsubsec{正在承担的与本项目相关的科研项目情况}{(申请人正在承担的与本项目相关的科研项目情况,包括国家自然科学基金的项目和国家其他科技计划项目,要注明项目的资助机构、项目类别、批准号、项目名称、\hspace{-4pt}获资助金额、\hspace{-4pt}起止年月、\hspace{-4pt}与本项目的关系及负责的内容等);}
无

\NSFCsubsec{完成国家自然科学基金项目情况}{(对申请人负责的前一个已资助期满的科学基金项目(项目名称及批准号)完成情况、后续研究进展及与本申请项目的关系加以详细说明。另附该项目的研究工作总结摘要(限500字)和相关成果详细目录)。}

无


\NSFCsec{其他需要说明的情况}{}

\NSFCsubsec{}{申请人同年申请不同类型的国家自然科学基金项目情况(列明同年申请的其他项目的项目类型、项目名称信息,并说明与本项目之间的区别与联系;已收到自然科学基金委不予受理或不予资助决定的,无需列出)。}
无

\NSFCsubsec{}{~具有高级专业技术职务(职称)的申请人是否存在同年申请或者参与申请国家自然科学基金项目的单位不一致的情况;如存在上述情况,列明所涉及人员的姓名,申请或参与申请的其他项目的项目类型、项目名称、单位名称、上述人员在该项目中是申请人还是参与者,并说明单位不一致原因。}
无

\NSFCsubsec{}{~具有高级专业技术职务(职称)的申请人是否存在与正在承担的国家自然科学基金项目的单位不一致的情况;如存在上述情况,列明所涉及人员的姓名,正在承担项目的批准号、项目类型、项目名称、单位名称、起止年月,并说明单位不一致原因。}
无

\NSFCsubsec{}{~同年以不同专业技术职务(职称)申请或参与申请科学基金项目的情况(应详细说明原因)。}
无

\NSFCsubsec{}{其他。}
无

\end{document}